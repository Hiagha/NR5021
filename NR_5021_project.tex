\documentclass[]{article}
\usepackage{lmodern}
\usepackage{amssymb,amsmath}
\usepackage{ifxetex,ifluatex}
\usepackage{fixltx2e} % provides \textsubscript
\ifnum 0\ifxetex 1\fi\ifluatex 1\fi=0 % if pdftex
  \usepackage[T1]{fontenc}
  \usepackage[utf8]{inputenc}
\else % if luatex or xelatex
  \ifxetex
    \usepackage{mathspec}
  \else
    \usepackage{fontspec}
  \fi
  \defaultfontfeatures{Ligatures=TeX,Scale=MatchLowercase}
\fi
% use upquote if available, for straight quotes in verbatim environments
\IfFileExists{upquote.sty}{\usepackage{upquote}}{}
% use microtype if available
\IfFileExists{microtype.sty}{%
\usepackage{microtype}
\UseMicrotypeSet[protrusion]{basicmath} % disable protrusion for tt fonts
}{}
\usepackage[margin=1in]{geometry}
\usepackage{hyperref}
\hypersetup{unicode=true,
            pdftitle={NR\_5021\_project},
            pdfborder={0 0 0},
            breaklinks=true}
\urlstyle{same}  % don't use monospace font for urls
\usepackage{color}
\usepackage{fancyvrb}
\newcommand{\VerbBar}{|}
\newcommand{\VERB}{\Verb[commandchars=\\\{\}]}
\DefineVerbatimEnvironment{Highlighting}{Verbatim}{commandchars=\\\{\}}
% Add ',fontsize=\small' for more characters per line
\usepackage{framed}
\definecolor{shadecolor}{RGB}{248,248,248}
\newenvironment{Shaded}{\begin{snugshade}}{\end{snugshade}}
\newcommand{\KeywordTok}[1]{\textcolor[rgb]{0.13,0.29,0.53}{\textbf{#1}}}
\newcommand{\DataTypeTok}[1]{\textcolor[rgb]{0.13,0.29,0.53}{#1}}
\newcommand{\DecValTok}[1]{\textcolor[rgb]{0.00,0.00,0.81}{#1}}
\newcommand{\BaseNTok}[1]{\textcolor[rgb]{0.00,0.00,0.81}{#1}}
\newcommand{\FloatTok}[1]{\textcolor[rgb]{0.00,0.00,0.81}{#1}}
\newcommand{\ConstantTok}[1]{\textcolor[rgb]{0.00,0.00,0.00}{#1}}
\newcommand{\CharTok}[1]{\textcolor[rgb]{0.31,0.60,0.02}{#1}}
\newcommand{\SpecialCharTok}[1]{\textcolor[rgb]{0.00,0.00,0.00}{#1}}
\newcommand{\StringTok}[1]{\textcolor[rgb]{0.31,0.60,0.02}{#1}}
\newcommand{\VerbatimStringTok}[1]{\textcolor[rgb]{0.31,0.60,0.02}{#1}}
\newcommand{\SpecialStringTok}[1]{\textcolor[rgb]{0.31,0.60,0.02}{#1}}
\newcommand{\ImportTok}[1]{#1}
\newcommand{\CommentTok}[1]{\textcolor[rgb]{0.56,0.35,0.01}{\textit{#1}}}
\newcommand{\DocumentationTok}[1]{\textcolor[rgb]{0.56,0.35,0.01}{\textbf{\textit{#1}}}}
\newcommand{\AnnotationTok}[1]{\textcolor[rgb]{0.56,0.35,0.01}{\textbf{\textit{#1}}}}
\newcommand{\CommentVarTok}[1]{\textcolor[rgb]{0.56,0.35,0.01}{\textbf{\textit{#1}}}}
\newcommand{\OtherTok}[1]{\textcolor[rgb]{0.56,0.35,0.01}{#1}}
\newcommand{\FunctionTok}[1]{\textcolor[rgb]{0.00,0.00,0.00}{#1}}
\newcommand{\VariableTok}[1]{\textcolor[rgb]{0.00,0.00,0.00}{#1}}
\newcommand{\ControlFlowTok}[1]{\textcolor[rgb]{0.13,0.29,0.53}{\textbf{#1}}}
\newcommand{\OperatorTok}[1]{\textcolor[rgb]{0.81,0.36,0.00}{\textbf{#1}}}
\newcommand{\BuiltInTok}[1]{#1}
\newcommand{\ExtensionTok}[1]{#1}
\newcommand{\PreprocessorTok}[1]{\textcolor[rgb]{0.56,0.35,0.01}{\textit{#1}}}
\newcommand{\AttributeTok}[1]{\textcolor[rgb]{0.77,0.63,0.00}{#1}}
\newcommand{\RegionMarkerTok}[1]{#1}
\newcommand{\InformationTok}[1]{\textcolor[rgb]{0.56,0.35,0.01}{\textbf{\textit{#1}}}}
\newcommand{\WarningTok}[1]{\textcolor[rgb]{0.56,0.35,0.01}{\textbf{\textit{#1}}}}
\newcommand{\AlertTok}[1]{\textcolor[rgb]{0.94,0.16,0.16}{#1}}
\newcommand{\ErrorTok}[1]{\textcolor[rgb]{0.64,0.00,0.00}{\textbf{#1}}}
\newcommand{\NormalTok}[1]{#1}
\usepackage{graphicx,grffile}
\makeatletter
\def\maxwidth{\ifdim\Gin@nat@width>\linewidth\linewidth\else\Gin@nat@width\fi}
\def\maxheight{\ifdim\Gin@nat@height>\textheight\textheight\else\Gin@nat@height\fi}
\makeatother
% Scale images if necessary, so that they will not overflow the page
% margins by default, and it is still possible to overwrite the defaults
% using explicit options in \includegraphics[width, height, ...]{}
\setkeys{Gin}{width=\maxwidth,height=\maxheight,keepaspectratio}
\IfFileExists{parskip.sty}{%
\usepackage{parskip}
}{% else
\setlength{\parindent}{0pt}
\setlength{\parskip}{6pt plus 2pt minus 1pt}
}
\setlength{\emergencystretch}{3em}  % prevent overfull lines
\providecommand{\tightlist}{%
  \setlength{\itemsep}{0pt}\setlength{\parskip}{0pt}}
\setcounter{secnumdepth}{0}
% Redefines (sub)paragraphs to behave more like sections
\ifx\paragraph\undefined\else
\let\oldparagraph\paragraph
\renewcommand{\paragraph}[1]{\oldparagraph{#1}\mbox{}}
\fi
\ifx\subparagraph\undefined\else
\let\oldsubparagraph\subparagraph
\renewcommand{\subparagraph}[1]{\oldsubparagraph{#1}\mbox{}}
\fi

%%% Use protect on footnotes to avoid problems with footnotes in titles
\let\rmarkdownfootnote\footnote%
\def\footnote{\protect\rmarkdownfootnote}

%%% Change title format to be more compact
\usepackage{titling}

% Create subtitle command for use in maketitle
\newcommand{\subtitle}[1]{
  \posttitle{
    \begin{center}\large#1\end{center}
    }
}

\setlength{\droptitle}{-2em}

  \title{NR\_5021\_project}
    \pretitle{\vspace{\droptitle}\centering\huge}
  \posttitle{\par}
    \author{}
    \preauthor{}\postauthor{}
      \predate{\centering\large\emph}
  \postdate{\par}
    \date{10/3/2018}


\begin{document}
\maketitle

\section{Better Know an R Package}\label{better-know-an-r-package}

\subsection{R/Cowplot}\label{rcowplot}

Cowplot is a package developed to work in tandem with R/ggplot2. From
the cowplot cran page, ``The cowplot package is a simple add-on to
ggplot2. It is meant to provide a publication-ready theme for ggplot2,
one that requires a minimum amount of fiddling with sizes of axis
labels, plot backgrounds, etc.''

\begin{Shaded}
\begin{Highlighting}[]
\CommentTok{#install.packages("cowplot")}
\KeywordTok{library}\NormalTok{(tidyverse); }\KeywordTok{library}\NormalTok{(ggplot2)}
\end{Highlighting}
\end{Shaded}

\subsubsection{Load in the data}\label{load-in-the-data}

We are using a data set that is within ggplot2 as well as a dataset
collected from class responses. The data set from ggplot2 is called
``msleep'' and records average sleep data, taxonomy, diet information,
and brain and body weight for 83 mammal species. The class data includes
reported hours of sleep and body weight in kilograms.

\begin{Shaded}
\begin{Highlighting}[]
\NormalTok{animal <-}\StringTok{ }\NormalTok{msleep}

\NormalTok{weight <-}\StringTok{ }\KeywordTok{c}\NormalTok{(}\DecValTok{74}\NormalTok{,}\DecValTok{63}\NormalTok{,}\DecValTok{112}\NormalTok{,}\DecValTok{70}\NormalTok{,}\DecValTok{55}\NormalTok{,}\DecValTok{50}\NormalTok{,}\DecValTok{75}\NormalTok{,}\DecValTok{59}\NormalTok{,}\DecValTok{93}\NormalTok{,}\FloatTok{88.64}\NormalTok{,}\FloatTok{54.4}\NormalTok{,}\DecValTok{84}\NormalTok{,}\FloatTok{68.18}\NormalTok{,}\DecValTok{130}\NormalTok{,}\DecValTok{100}\NormalTok{,}\FloatTok{79.54}\NormalTok{,}\DecValTok{77}\NormalTok{,}\DecValTok{52}\NormalTok{,}\FloatTok{79.5}\NormalTok{,}\FloatTok{68.2}\NormalTok{,}\FloatTok{57.72}\NormalTok{)}

\NormalTok{sleep<-}\StringTok{ }\KeywordTok{c}\NormalTok{(}\DecValTok{6}\NormalTok{,}\DecValTok{6}\NormalTok{,}\DecValTok{6}\NormalTok{,}\FloatTok{6.5}\NormalTok{,}\FloatTok{6.5}\NormalTok{,}\DecValTok{7}\NormalTok{,}\DecValTok{7}\NormalTok{,}\DecValTok{7}\NormalTok{,}\DecValTok{7}\NormalTok{,}\DecValTok{7}\NormalTok{,}\DecValTok{7}\NormalTok{,}\DecValTok{7}\NormalTok{,}\DecValTok{7}\NormalTok{,}\DecValTok{7}\NormalTok{,}\FloatTok{7.25}\NormalTok{,}\FloatTok{7.5}\NormalTok{,}\FloatTok{7.5}\NormalTok{,}\DecValTok{8}\NormalTok{,}\DecValTok{8}\NormalTok{,}\DecValTok{8}\NormalTok{,}\DecValTok{9}\NormalTok{)}

\NormalTok{class_sleep <-}\StringTok{ }\KeywordTok{data.frame}\NormalTok{(weight,sleep)}

\NormalTok{class_sleep_summary<-}\StringTok{ }\KeywordTok{summarize}\NormalTok{(class_sleep,}
                                \DataTypeTok{mean.sleep =} \KeywordTok{mean}\NormalTok{(sleep),}
                                \DataTypeTok{mean.weight =} \KeywordTok{mean}\NormalTok{(weight))}
\NormalTok{class_sleep_summary}
\end{Highlighting}
\end{Shaded}

\begin{verbatim}
##   mean.sleep mean.weight
## 1   7.107143    75.72286
\end{verbatim}

\begin{Shaded}
\begin{Highlighting}[]
\NormalTok{## Now we will add our class data to the msleep dataset. }
\NormalTok{name<-}\StringTok{ "Class"}
\NormalTok{class.sleep<-}\StringTok{ }\KeywordTok{cbind}\NormalTok{(class_sleep_summary,name)}

\NormalTok{class.sleep<-}\StringTok{ }\KeywordTok{data.frame}\NormalTok{(class.sleep}\OperatorTok{$}\NormalTok{name,class.sleep}\OperatorTok{$}\NormalTok{mean.sleep,class.sleep}\OperatorTok{$}\NormalTok{mean.weight)}
\KeywordTok{names}\NormalTok{(class.sleep) <-}\StringTok{ }\KeywordTok{c}\NormalTok{(}\StringTok{"msleep.name"}\NormalTok{, }\StringTok{"msleep.sleep_total"}\NormalTok{, }\StringTok{"msleep.bodywt"}\NormalTok{)}
\NormalTok{new.msleep <-}\StringTok{ }\KeywordTok{data.frame}\NormalTok{(msleep}\OperatorTok{$}\NormalTok{name, msleep}\OperatorTok{$}\NormalTok{sleep_total, msleep}\OperatorTok{$}\NormalTok{bodywt)}

\NormalTok{new.msleep.class<-}\StringTok{ }\KeywordTok{rbind}\NormalTok{(new.msleep, class.sleep)}
\end{Highlighting}
\end{Shaded}

\subsection{ggplot2}\label{ggplot2}

First we will make several different plots using solely ggplot2
functions

\begin{Shaded}
\begin{Highlighting}[]
\NormalTok{msleep.hist <-}\StringTok{ }\KeywordTok{ggplot}\NormalTok{(new.msleep.class, }\KeywordTok{aes}\NormalTok{(}\DataTypeTok{x=}\NormalTok{msleep.sleep_total)) }\OperatorTok{+}\StringTok{ }\KeywordTok{geom_histogram}\NormalTok{(}\DataTypeTok{binwidth=}\DecValTok{1}\NormalTok{, }\KeywordTok{aes}\NormalTok{(}\DataTypeTok{fill=}\NormalTok{..count..)) }\OperatorTok{+}\StringTok{ }\KeywordTok{labs}\NormalTok{(}\DataTypeTok{title=}\StringTok{"Mammal and Class Sleep"}\NormalTok{) }\OperatorTok{+}\StringTok{ }\KeywordTok{xlab}\NormalTok{(}\StringTok{"Total Hours of Sleep"}\NormalTok{)}
\NormalTok{msleep.hist }\CommentTok{#Creates a histogram for hours of sleep for both the msleep and the class data}
\end{Highlighting}
\end{Shaded}

\includegraphics{NR_5021_project_files/figure-latex/ggplot-1.pdf}

\begin{Shaded}
\begin{Highlighting}[]
\NormalTok{p <-}\StringTok{ }\KeywordTok{ggplot}\NormalTok{(animal, }\KeywordTok{aes}\NormalTok{(}\DataTypeTok{x =}\NormalTok{ order, }\DataTypeTok{y =}\NormalTok{ sleep_total)) }\OperatorTok{+}\StringTok{ }\KeywordTok{theme}\NormalTok{(}\DataTypeTok{axis.text.x =} \KeywordTok{element_text}\NormalTok{(}\DataTypeTok{face =} \StringTok{"italic"}\NormalTok{, }\DataTypeTok{angle =} \DecValTok{90}\NormalTok{, }\DataTypeTok{vjust =} \FloatTok{0.5}\NormalTok{)) }\OperatorTok{+}\StringTok{ }\KeywordTok{geom_point}\NormalTok{() }\OperatorTok{+}\StringTok{ }\KeywordTok{aes}\NormalTok{(}\DataTypeTok{colour =} \KeywordTok{log}\NormalTok{(bodywt))}
\NormalTok{p }
\end{Highlighting}
\end{Shaded}

\includegraphics{NR_5021_project_files/figure-latex/ggplot-2.pdf}

\begin{Shaded}
\begin{Highlighting}[]
\NormalTok{sleep.plot<-}\KeywordTok{ggplot}\NormalTok{(msleep, }\KeywordTok{aes}\NormalTok{(}\DataTypeTok{x=}\NormalTok{msleep}\OperatorTok{$}\NormalTok{sleep_total, }\DataTypeTok{y=}\NormalTok{msleep}\OperatorTok{$}\NormalTok{bodywt,}\DataTypeTok{colour=}\NormalTok{order)) }\OperatorTok{+}\StringTok{ }\KeywordTok{geom_point}\NormalTok{() }\OperatorTok{+}\StringTok{ }\KeywordTok{labs}\NormalTok{(}\DataTypeTok{x=}\StringTok{"Mean sleep (hours)"}\NormalTok{, }\DataTypeTok{y=} \StringTok{"Mean body weight (kg)"}\NormalTok{)}
\NormalTok{sleep.plot}
\end{Highlighting}
\end{Shaded}

\includegraphics{NR_5021_project_files/figure-latex/ggplot-3.pdf} \#\#
The cowplot difference Notice that in cowplot, the default gray
background and grid lines present in ggplot2 are missing. A similar
design can also be achieved in ggplot using ``+ theme\_classic''

One unique feature of cowplot is the capacity to add images to your
plots. For instance, we can add an image of an elephant to the
background. To include this feature, we must first download the package
``magick.'' You can change the size of the background image using
``scale=''. Before we can use cowplot, we still have to have a basic
understanding of ggplot

\begin{Shaded}
\begin{Highlighting}[]
\KeywordTok{library}\NormalTok{(cowplot)}
\end{Highlighting}
\end{Shaded}

\begin{verbatim}
## 
## Attaching package: 'cowplot'
\end{verbatim}

\begin{verbatim}
## The following object is masked from 'package:ggplot2':
## 
##     ggsave
\end{verbatim}

\begin{Shaded}
\begin{Highlighting}[]
\CommentTok{#install.packages("magick")}
\KeywordTok{library}\NormalTok{(magick)}
\end{Highlighting}
\end{Shaded}

\begin{verbatim}
## Linking to ImageMagick 6.9.9.14
## Enabled features: cairo, freetype, fftw, ghostscript, lcms, pango, rsvg, webp
## Disabled features: fontconfig, x11
\end{verbatim}

\begin{Shaded}
\begin{Highlighting}[]
\NormalTok{elephant <-}\StringTok{ }\KeywordTok{image_read}\NormalTok{(}\StringTok{'https://github.com/Hiagha/NR5021/blob/master/elephant.png?raw=true'}\NormalTok{)}
\NormalTok{elephant.background <-}\StringTok{ }\KeywordTok{ggdraw}\NormalTok{() }\OperatorTok{+}\StringTok{ }\KeywordTok{draw_image}\NormalTok{(elephant,}\DataTypeTok{scale=}\FloatTok{0.8}\NormalTok{) }\OperatorTok{+}\StringTok{ }\KeywordTok{draw_plot}\NormalTok{(sleep.plot)}
\NormalTok{elephant.background}
\end{Highlighting}
\end{Shaded}

\includegraphics{NR_5021_project_files/figure-latex/unnamed-chunk-1-1.pdf}
\#\# Labeling You can also use cowplot to add labels to plots. The x and
y axis denote where you want the label to appear on the plot. One
limitation of cowplot is that the code can get long as more features,
like labels, are added to the plot

\begin{Shaded}
\begin{Highlighting}[]
\NormalTok{elephant.background}\OperatorTok{+}\KeywordTok{draw_label}\NormalTok{(}\StringTok{"Image publically available from pngimg.com"}\NormalTok{, }\DataTypeTok{x=}\FloatTok{0.5}\NormalTok{, }\DataTypeTok{y=}\FloatTok{0.9}\NormalTok{, }\DataTypeTok{size=}\DecValTok{12}\NormalTok{, }\DataTypeTok{fontface =} \StringTok{"italic"}\NormalTok{)}
\end{Highlighting}
\end{Shaded}

\includegraphics{NR_5021_project_files/figure-latex/unnamed-chunk-2-1.pdf}

\begin{Shaded}
\begin{Highlighting}[]
\NormalTok{##Let's add a label to our histogram to denote the location of the mean class sleep total with respect to all mammal sleep totals.}

\NormalTok{sleep.hist<-}\StringTok{ }\NormalTok{msleep.hist }\OperatorTok{+}\StringTok{ }\KeywordTok{draw_label}\NormalTok{(}\StringTok{"(*) = Class mean sleep (7.1 hours)"}\NormalTok{, }\DataTypeTok{x=}\DecValTok{15}\NormalTok{, }\DataTypeTok{y=}\DecValTok{10}\NormalTok{, }\DataTypeTok{size=}\DecValTok{10}\NormalTok{,}\DataTypeTok{fontface=}\StringTok{"plain"}\NormalTok{) }\OperatorTok{+}\StringTok{ }\KeywordTok{draw_label}\NormalTok{(}\StringTok{"*"}\NormalTok{, }\DataTypeTok{x=}\FloatTok{7.1}\NormalTok{,}\DataTypeTok{y=}\FloatTok{2.1}\NormalTok{,}\DataTypeTok{size=}\DecValTok{20}\NormalTok{,}\DataTypeTok{fontface=}\StringTok{"bold"}\NormalTok{)}
\NormalTok{sleep.hist}
\end{Highlighting}
\end{Shaded}

\includegraphics{NR_5021_project_files/figure-latex/unnamed-chunk-2-2.pdf}

\subsection{Comparing plots}\label{comparing-plots}

Say we wanted to look at the animal sleep data set without the elephant
data points, which are the mammals with the two largest masses in the
data set. We can create a new data set without these points and plot the
Mass vs.~Mean Sleep as before
\textless{}\textless{}\textless{}\textless{}\textless{}\textless{}\textless{}
HEAD

\begin{Shaded}
\begin{Highlighting}[]
\NormalTok{no.elephant.animal <-}\StringTok{ }\NormalTok{animal[}\OperatorTok{-}\KeywordTok{c}\NormalTok{(}\DecValTok{21}\NormalTok{,}\DecValTok{36}\NormalTok{),] }\CommentTok{#This code gets rid of rows 21 and 36 in the animal data set, corresponding to the two elephants (African and Asian)}
\CommentTok{#View(no.elephant.animal) #This code lets us view the new data table to confirm that the rows with the elephant information are eliminated}
\NormalTok{no.elephant.sleep.plot <-}\StringTok{ }\KeywordTok{ggplot}\NormalTok{(no.elephant.animal, }\KeywordTok{aes}\NormalTok{(}\DataTypeTok{x=}\NormalTok{no.elephant.animal}\OperatorTok{$}\NormalTok{sleep_total, }\DataTypeTok{y=}\NormalTok{no.elephant.animal}\OperatorTok{$}\NormalTok{bodywt,}\DataTypeTok{colour=}\NormalTok{order)) }\OperatorTok{+}\StringTok{ }\KeywordTok{geom_point}\NormalTok{() }\OperatorTok{+}\StringTok{ }\KeywordTok{labs}\NormalTok{(}\DataTypeTok{x=}\StringTok{"Mean sleep (hours)"}\NormalTok{, }\DataTypeTok{y=} \StringTok{"Mean body weight (kg)"}\NormalTok{)}
\NormalTok{no.elephant.sleep.plot}
\end{Highlighting}
\end{Shaded}

\includegraphics{NR_5021_project_files/figure-latex/unnamed-chunk-3-1.pdf}
\#\# Combining plots Another advantage of cowplot is that it can combine
plots into a grid or other arrangements. We can specify the size of each
plot and where we want it to appear. For example, we can insert the
smaller mammal mean sleep vs.~mean body mass plot by a larger version of
the same plot that excludes the elephant data points. To do this, we
will again use the ggdraw() and draw\_plot functions of cowplot. We will
also remove the legend for the smaller plot and place the legend on the
left for the larger plot to make it easier to view both graphs. One
difficulty with combining plots into layouts besides grid arrangements
is it can be challenging to anticipate the coordinates where you want
the plot to appear. Achieving the layout that you want may take some
trial and error in the code for the coordinates. Here, the first two
numbers denote where you want the plot on the x and y axis. The second
set of coordinates show the scales of the x and y axis, respectively

\begin{Shaded}
\begin{Highlighting}[]
\KeywordTok{ggdraw}\NormalTok{() }\OperatorTok{+}\StringTok{ }\KeywordTok{draw_plot}\NormalTok{(no.elephant.sleep.plot }\OperatorTok{+}\StringTok{ }\KeywordTok{theme}\NormalTok{(}\DataTypeTok{legend.position =} \StringTok{"left"}\NormalTok{)) }\OperatorTok{+}\StringTok{ }\KeywordTok{draw_plot}\NormalTok{(sleep.plot }\OperatorTok{+}\StringTok{ }\KeywordTok{theme}\NormalTok{(}\DataTypeTok{legend.position =} \StringTok{"none"}\NormalTok{), }\FloatTok{0.6}\NormalTok{,}\FloatTok{0.5}\NormalTok{,}\FloatTok{0.4}\NormalTok{,}\FloatTok{0.4}\NormalTok{)}
\end{Highlighting}
\end{Shaded}

\includegraphics{NR_5021_project_files/figure-latex/unnamed-chunk-4-1.pdf}

\subsection{Presentation ready plots}\label{presentation-ready-plots}

Cowplot also allows for multiple plots to put plotted gridwise with
labels and then saved as a single object. This is very useful when
getting plots ready for presentation or publication. Cowplot improves
the functionality of \emph{ggsave( )} and makes it possible to save
these combined plots as a single object.

\begin{Shaded}
\begin{Highlighting}[]
\NormalTok{side.by.side <-}\StringTok{ }\KeywordTok{plot_grid}\NormalTok{(p, msleep.hist, }\DataTypeTok{labels =} \KeywordTok{c}\NormalTok{(}\StringTok{"A"}\NormalTok{, }\StringTok{"B"}\NormalTok{), }\DataTypeTok{align =} \StringTok{"hv"}\NormalTok{, }\DataTypeTok{axis =} \StringTok{"tb"}\NormalTok{)}
\NormalTok{side.by.side}
\end{Highlighting}
\end{Shaded}

\includegraphics{NR_5021_project_files/figure-latex/unnamed-chunk-5-1.pdf}

\begin{Shaded}
\begin{Highlighting}[]
\CommentTok{# We can also save our combined plots using ggsave(), setting appropriate height/width and DPI for whatever project you are working on.}

\NormalTok{four.plot <-}\StringTok{ }\KeywordTok{plot_grid}\NormalTok{(sleep.plot, no.elephant.sleep.plot, p, sleep.hist, }\DataTypeTok{labels =} \KeywordTok{c}\NormalTok{(}\StringTok{"A"}\NormalTok{, }\StringTok{"B"}\NormalTok{, }\StringTok{"C"}\NormalTok{, }\StringTok{"D"}\NormalTok{), }\DataTypeTok{align =} \StringTok{"hv"}\NormalTok{, }\DataTypeTok{axis =} \StringTok{"tblr"}\NormalTok{)}
\KeywordTok{ggsave}\NormalTok{(}\StringTok{"fourplot.png"}\NormalTok{, four.plot, }\DataTypeTok{width =} \DecValTok{10}\NormalTok{, }\DataTypeTok{height =} \DecValTok{12}\NormalTok{)}

\NormalTok{plots <-}\StringTok{ }\KeywordTok{image_read}\NormalTok{(}\StringTok{"fourplot.png"}\NormalTok{)}
\KeywordTok{ggdraw}\NormalTok{() }\OperatorTok{+}\StringTok{ }\KeywordTok{draw_image}\NormalTok{(plots)}
\end{Highlighting}
\end{Shaded}

\includegraphics{NR_5021_project_files/figure-latex/unnamed-chunk-5-2.pdf}

\subsection{Contributions}\label{contributions}

James created the class survey, combined the class data with the msleep
dataset and created the histograms for the combined datasets. Hannah
created the plots of sleep vs bodyweight associated with the msleep
dataset and highlighted the cowplot properties of adding photos to a
figure. She also highlighted the limitations and difficulties of cowplot
throughout the Rmarkdown document. Husain provided the cowplot expertise
and highlighted the cowplot functions of adding a subset image within a
figure, aligning various plots and ensuring the overall figure is
publication-ready. Clare was responsible for the Flipgrid video
highlighting the cowplot package and its benefits and limitations.


\end{document}
